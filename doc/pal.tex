\documentclass{report}
\usepackage{fancyhdr}
\usepackage{setspace}
\usepackage{lipsum}
\pagestyle{fancy}
\lhead{CMPUT 415}
\chead{pal Compiler}
\rhead{Fall 2012}
\lfoot{Group 5}
\cfoot{}
\rfoot{\thepage}
\setlength{\parindent}{0pt}
\doublespacing

\title{CMPUT 415 - Project Checkpoint 1\\Program Documentation}
\author{Mike Bujold \\
Dan Chui \\ 
James Osgood \\
Paul Vandermeer}

\begin{document}
\maketitle
\textbf{The man page is available by typing} \emph{make man}

\section*{Introduction}
The goals for this checkpoint are to have a functional compiler 'shell'. That is, it must be able to parse a program listing and report on any errors in the program in a way that is meaningful to the user so that corrective action may be taken. It must also recover from each error in such a manner that further reporting of errors is possible, up until the end of the program listing.


\section*{Handling of Lexical Units}
Our lexer parses tokens in the following order: first, all single and multi-line comments are parsed, followed by reserved \emph{pal} keywords, then numbers and variable names. Relational operators and other lexical units are parsed last. Any remaining unparsed tokens are presumed invalid, and the grammar handles reporting the error accordingly.

\section*{Syntax Error Reporting \& Recovery}
We created myerror.c to catch all the errors. When an error is found, it is added to a linked list. As soon as we finish parsing a line, we output that list, clear it, and move onto the next line. 
\lipsum[4-5]

\section*{Problems}
\lipsum[6]

\end{document}